\documentclass[11pt]{article}
\usepackage{amsmath}
\title{Algorithms Exercises Answers and Notes}
\author{exceedhl@gmail.com}

\begin{document}
\maketitle

\section{Chapter 1}

\subsection{RSA algorithm}

$p$, $q$ are primes, $N=pq$, $N'=(p-1)(q-1)$, $e$ is relatively prime
to $N'$. $x$ is the plain text and $x \in {0, 1, \ldots, N-1}$.

$N$ and $e$ are public, the encrypted text is:
\begin{equation}
  x' = x^e\ rem\ N, \quad
  x' \in {0, 1, \ldots, N-1}
\end{equation}

Let 
\begin{equation}
  de \equiv 1 \ modulo\ N', \quad
\end{equation}
$d$ can be calculated using extended Euclid algorithm.

Then
\begin{align}
  & de - 1 = kN'\\
  & de = kN' + 1\\
  & x'^d-x = x^{ed} - x = x^{1+k(p-1)(q-1)}-x
\end{align}

Because $p$, $q$ are primes and $x < N$, so according to Fermat's
little theorem:
\begin{align}
  & x^{p-1} \equiv 1\ modulo\ p\\
  \Rightarrow & x^{(p-1)k(q-1)} \equiv 1\ modulo\ p\\  
  & x^{q-1} \equiv 1\ modulo\ q\\
  \Rightarrow & x^{(q-1)k(p-1)} \equiv 1\ modulo\ q\\
  \Rightarrow & x^{(q-1)k(p-1)}-1\ rem\ pq = 0\\
  \Rightarrow & x^{(q-1)k(p-1)} \equiv 1\ modulo\ pq\\
  \Rightarrow & x'^d-x = x^{ed} - x = x^{1+k(p-1)(q-1)}-x\ rem\ pq = 0\\
  \Rightarrow & x'^d \equiv 1\ modulo\ N
\end{align}

Because $x^{ed} \equiv x\ modulo\ N$ and $x < N$, we can get $x$ by
calculating $x'^d\ rem\ N$.

So if we make $N$, $e$ public, keep $d$, $p$, $q$ secret, then we can
encrypt and decrypt messages using power and rem operations.

Finally, since $x^e\ rem\ N = x'$, and $x'^d\ rem\ N = x$, it is
obvious that they are bijection functions. 

\subsection{Exercises}

\subsubsection{1.1}
Consider the biggest possible value of adding three digits in base b:
$3(b-1)$. The biggest value of two digits in base b is: $b^2-1$. Now
we just need to prove $b^2-1 \ge 3(b-1)$.

\begin{align}
  b^2-1-3(b-1) & = b^2-3b+2 \\
  & = (b-1)(b-2) \\
  & \ge 0\quad (\forall\ b \ge 2)
\end{align}


\end{document}



