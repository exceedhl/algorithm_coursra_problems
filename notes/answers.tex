\documentclass[11pt]{article}
\usepackage{amsmath}
\usepackage{amsfonts}
\usepackage{amssymb}
\title{Algorithms Exercises Answers and Notes}
\author{exceedhl@gmail.com}

\newtheorem{thm}{Theorem}
\newcommand{\rem}{\ rem\ }

\begin{document}
\maketitle

\section{Chapter 1}

\subsection{Basic facts about rem}

\begin{thm}
  Rem operation has the following features: ($a, b \in \mathbb{Z}$)
  \begin{align}
    ab \rem N & = a(b \rem N) \rem N \\
    & = (a \rem N)b \rem N \\
    & = (a \rem N)(b \rem N) \rem N \\
    a^b \rem N & = (a \rem N)^b \\
    (a \pm b) \rem N & = ((a \rem N ) \pm (b \rem N)) \rem N
  \end{align}
\end{thm}
These features are important for solving $a^b \rem N$ problems.

\begin{thm}
  Suppose $a \equiv b \mod n$ and $c \equiv d \mod n$. Then 
  \begin{enumerate}
  \item $a+c \equiv b+d \mod n$
  \item $ac \equiv bd \mod n$
  \item $f(a) \equiv f(b) \mod n$ for any polynomial f(x) with integer coefficients.
  \end{enumerate}
\end{thm}

\begin{thm}
  About mod operation:
  \begin{align}
    & a \equiv b \mod N, b \equiv c \mod N \\
    \Rightarrow & a \equiv b \equiv c \mod N
  \end{align}

  If $p$ is a prime, $a$ is not a multiplicative of $p$, then 
  \begin{align}
    & a^{p-1} \equiv 1 \mod p \\
    \Rightarrow & a^n \rem p = a^{n \rem (p-1)} \rem p \label{eqn:primitive-roots}\\
    & a^m \equiv 1 \mod p \\
    \Rightarrow & a^n \rem p = a^{n \rem m} \rem p
  \end{align}
\end{thm}
\eqref{eqn:primitive-roots} is the key to solve $a^{b^c} \rem p$
problems. 

\begin{thm}
  If a has a multiplicative inverse modulo N , then this inverse is
  unique (modulo N).
\end{thm}

\begin{thm}
   If a has an inverse modulo b, then b has an inverse modulo a.
\end{thm}

\subsection{Theorems}

\begin{thm}
  If $p$, $q$ are primes, and $a$ is not a multiplier of either of
  them then we have:
  \begin{equation}
    a^{(p-1)(q-1)}\equiv 1\ mod\ pq
  \end{equation}
\end{thm}
This is useful in the proof of RSA algorithm.

\begin{thm}
  If $p$ is a prime, $a$ is an integer, then $GCD(a, p^n) \neq 1$ ($n
  > 0$ is a integer) if and only if $a=kp$ ($k$ is an integer).
\end{thm}

\begin{thm}
  If $p$ is a prime and $a<p$, then there exists $b<p$, so that
  $ab\equiv 1\ mod\ p$.
\end{thm}
\textit{Proof.} According to Fermat's theorem: there exists a
$a^{-1}$ so that $aa^{-1}\equiv 1\ mod\ p$.

Because $aa^{-1}\ rem\ p = a(a^{-1}\ rem\ p)\ rem\ p$ so ethat there
exists a $b<p$ so that $ab\equiv 1\ mod\ p$.

This theorem can be used to prove Wilson's theorem:
\begin{align}
  (p-1)! \equiv -1 \mod p
\end{align}

\begin{thm}
  For any two adjacent Fibonacci numbers $F_n$ and $F_{n+1}$,
  $GCD(F_{n}, F_{n+1}) = 1$.
\end{thm}


\subsection{RSA algorithm}

$p$, $q$ are primes, $N=pq$, $N'=(p-1)(q-1)$, $e$ is relatively prime
to $N'$. $x$ is the plain text and $x \in {0, 1, \ldots, N-1}$.

$N$ and $e$ are public, the encrypted text is:
\begin{equation}
  x' = x^e\ rem\ N, \quad
  x' \in {0, 1, \ldots, N-1}
\end{equation}

Let 
\begin{equation}
  de \equiv 1 \ mod\ N', \quad
\end{equation}
$d$ can be calculated using extended Euclid algorithm.

Then
\begin{align}
  & de - 1 = kN'\\
  & de = kN' + 1\\
  & x'^d-x = x^{ed} - x = x^{1+k(p-1)(q-1)}-x
\end{align}

Because $p$, $q$ are primes and $x < N$, so according to Fermat's
little theorem:
\begin{align}
  & x^{p-1} \equiv 1\ mod\ p\\
  \Rightarrow & x^{(p-1)k(q-1)} \equiv 1\ mod\ p\\  
  & x^{q-1} \equiv 1\ mod\ q\\
  \Rightarrow & x^{(q-1)k(p-1)} \equiv 1\ mod\ q\\
  \Rightarrow & x^{(q-1)k(p-1)}-1\ rem\ pq = 0\\
  \Rightarrow & x^{(q-1)k(p-1)} \equiv 1\ mod\ pq\\
  \Rightarrow & x'^d-x = x^{ed} - x = x^{1+k(p-1)(q-1)}-x\ rem\ pq = 0\\
  \Rightarrow & x'^d \equiv 1\ mod\ N
\end{align}

Because $x^{ed} \equiv x\ mod\ N$ and $x < N$, we can get $x$ by
calculating $x'^d\ rem\ N$.

So if we make $N$, $e$ public, keep $d$, $p$, $q$ secret, then we can
encrypt and decrypt messages using power and rem operations.

Finally, since $x^e\ rem\ N = x'$, and $x'^d\ rem\ N = x$, it is
obvious that they are bijection functions. 

\subsection{Exercises}

\subsubsection{e1.1}
Consider the biggest possible value of adding three digits in base b:
$3(b-1)$. The biggest value of two digits in base b is: $b^2-1$. Now
we just need to prove $b^2-1 \ge 3(b-1)$.

\begin{align}
  b^2-1-3(b-1) & = b^2-3b+2 \\
  & = (b-1)(b-2) \\
  & \ge 0\quad (\forall\ b \ge 2)
\end{align}

\subsubsection{e1.26}

According to Feramt's theorem, if $p$ is a prime and $k$ is not a
multiplier of $p$, then $k^{p-1} \equiv 1\ mod\ p$.

If there are two primes and $k$ is not multiplier of either one, then
we can have (refer to the proof of RSA algorithm):
\begin{equation}
  k^{(p-1)(q-1)} \equiv 1\ mod\ pq
\end{equation}

Let the least significant digit of $17^{17^{17}}$ is $d$:
\begin{align}
  17^{17^{17}} \equiv d\ mod\ 10
\end{align}

$10=2*5$ so set $p=2$, $q=5$, we just need to find $a$ which is not
multiplier of $2$ and $5$, then we have $a^4 \equiv 1\ mod\ 10$.

Continuously use this $a^4$ to divide the original number to get $d$.

Let $a=17$, so 
\begin{align}
  17^{17^{17}} & \equiv 17^{289} \\
  & \equiv 17*17^{288} \\
  & \equiv 17*(17^4)^{72} \\
  & \equiv 17 \\
  & \equiv d \ mod\ 10 \\
  & \Rightarrow\ d=7
\end{align}

\subsubsection{e1.29}

The problem assumes $x_1 < m,\ x_2 < m$.

(a) is universal hashing function, the proof is same with the IP
hashing function example.

(b) is not universal. Suppose $(x_1, x_2)$ is different with $(y_1,
y_2)$ and $x_2$ is different with $y_2$. $h_{a_1, a_2}(x_1, x_2)$
equals with $h_{a_1, a_2}(y_1, y_2)$ means:

\begin{align}
  & a_1x_1+a_2x_2 \equiv a_1y_1+a_2y_2 \ mod\ m \\
  & a_1(x_1-y_1) \equiv a_2(y_1-y_2) \ mod\ m \label{eq:1.29.1}
\end{align}

Suppose the left side equals to $c$, then if $(y_1-y_2)$ is relative
prime to $m$, then $a_2$ must be $c(y_1-y_2)^{-1}$.

Because $m=2^k$ is not prime and the number of numbers that are
relative prime to $m$ is $\phi(m) = \phi(2^k) = 2^k - 2^{k-1}$.

The chance of $(y_1-y_2)$ being relative prime to $m=2^k$ is $1/2$
because any odd number is relative prime to $2^k$, so the chance of
\eqref{eq:1.29.1} holding is: $1/2 * 1/(2^k-2^{k-1}) = 1/2^k$.

When $(y_1-y_2)$ is even there exists some $a_2$ to make
\eqref{eq:1.29.1} hold. For example:

\begin{align}
  m=2^3=8 \\
  y_1-y_2=2 \\
  c=2 \\
  2 \equiv a_2*2 \ mod\ 8 \\
  \Rightarrow a_2=5
\end{align}

So the overall probability of making \eqref{eq:1.29.1} hold is greater
than $1/2^k = 1/m$.

(c) is not universal. Take an arbitrary $f$, the probability of a number
$p$'s key being conflict with another's key is $1/(m-1) > 1/m$.

\subsubsection{1.32 Perfect square/power check}



\end{document}



